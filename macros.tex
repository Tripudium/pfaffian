%=================================================
% Packages
%=================================================

\usepackage{amsmath,amsfonts,amsbsy,amsgen,amscd,mathrsfs,amssymb,amscd,amsthm}
\usepackage[UKenglish]{babel}
\usepackage{eurosym}
\usepackage[dvipsnames]{xcolor}
\usepackage{graphicx}
\usepackage{xfrac}
\usepackage{faktor}
\usepackage{caption}
\usepackage{copyrightbox}
\usepackage{microtype}
\usepackage{enumitem}
\usepackage{multicol}
\usepackage{multirow}
\usepackage{booktabs,longtable,tabu}
\setlength{\tabulinesep}{1pt}
\usepackage{float}
\usepackage[many]{tcolorbox}
\usepackage{mathtools}

% Algorithms
\usepackage[]{algorithm2e}
\usepackage{listings}

% Fonts
\usepackage{times}
%\usepackage{fourier}
%\usepackage[no-math]{fontspec}
%\setmainfont{optima}
%\usepackage{charter}
\usepackage{bm}              % boldmath must be called after the package
\usepackage{CJKutf8}         % Chinese
\usepackage[T1]{fontenc}

% Bibliography
\usepackage{natbib}
\usepackage{bibentry}
\usepackage{chapterbib}

% Referencing
\usepackage{url}
\usepackage[colorlinks=true]{hyperref}
\hypersetup{urlcolor=Blue}
\hypersetup{citecolor=Black}
\hypersetup{linkcolor=dark-gray}

% tikz
\usepackage{tikz,tikz-3dplot,tikz-cd}
\usetikzlibrary{matrix,arrows,shapes,calc,3d}
\usepackage{pgfplots}
%\usepackage{neuralnetwork}

%=================================================
% Package options
%=================================================

%\epigraphfontsize{\small\itshape}
%\setlength\epigraphwidth{8cm}
%\setlength\epigraphrule{0pt}
%\epigraphfontsize{\small\itshape}

% tikz
\pgfplotsset{compat=newest}
\tdplotsetmaincoords{60}{110}
\pgfmathsetmacro{\rvec}{.8}
\pgfmathsetmacro{\thetavec}{30}
\pgfmathsetmacro{\phivec}{60}
\tikzset{mynode/.style = {
    % The shape
    circle,
    % The size
    minimum size=2pt,
    % The color
    draw=black, fill=black}
}
\tikzset{mynodew/.style = {
    % The shape
    circle,
    % The size
    minimum size=2pt,
    % The color
    draw=black, fill=white}
}
\tikzset{vertex/.style={shape=circle, % style for a vertex
                        minimum size=3pt,
                        fill=gray,
                        inner sep = 0pt}}  
                       
\usetikzlibrary{decorations.markings}

\tikzstyle myBG=[line width=3pt,opacity=1]
 
\tikzset{->-/.style={decoration={
  markings,
  mark=at position #1 with {\arrow{>}}},postaction={decorate}}}
  
\newcommand{\drawBG}[2]
{
  \draw[white,myBG]  (#1) -- (#2);
  \draw[black,very thick] (#1) -- (#2);
}

\tikzset{
    partial ellipse/.style args={#1:#2:#3}{
        insert path={+ (#1:#3) arc (#1:#2:#3)}
    }
}

\newcommand{\strip}[1]{%
\shadedraw[very thick,top color=white,bottom color=gray,rotate=#1]
 (0:2.8453) ++ (-30:1.5359) arc (60:0:2)
 -- ++  (90:5) arc (0:60:2) -- ++ (150:3) arc (60:120:2) 
 -- ++ (210:5) arc (120:60:2) -- cycle;}
\newcommand{\MoebiusStrip}[3]{%
\begin{scope} [transform shape]
    \strip{0}
    \strip{120}
    \strip{-120}
    \draw (-60:3.5) node[scale=6,rotate=30] {#1};
    \draw (180:3.5) node[scale=4,rotate=-90]{#3};
    % redraw the first strip after clipping
    \clip (-1.4,2.4)--(-.3,6.1)--(1.3,6.1)--(5.3,3.7)--(5.3,-2.7)--cycle;
    \strip{0}
    \draw (60:3.5) node [gray,xscale=-4,yscale=4,rotate=30]{#2};
\end{scope}}
\newcounter{ga} %counter
\setcounter{ga}{0}

% Define block styles
\tikzstyle{decision} = [diamond, draw, fill=blue!20, 
    text width=4.5em, text badly centered, node distance=3cm, inner sep=0pt]
\tikzstyle{block} = [rectangle, draw, fill=blue!20, 
    text width=5em, text centered, rounded corners, minimum height=4em]
\tikzstyle{line} = [draw, -latex']
\tikzstyle{cloud} = [draw, ellipse,fill=red!20, node distance=3cm,
    minimum height=2em]
    
% Define neural network styles
\tikzset{%
  every neuron/.style={
    circle,
    draw,
    minimum size=0.5cm
  },
  neuron missing/.style={
    draw=none, 
    scale=4,
    text height=0.333cm,
    execute at begin node=\color{black}$\vdots$
  },
}    
    
% listing options
\tcbuselibrary{listings}
\lstloadlanguages{Python} %use listings with Python
\newlength\inwd
\setlength\inwd{1.7cm}
\newcounter{ipythcntr}
\newtcblisting{ipythonnb}[1][\theipythcntr]{
  enlarge left by=\inwd,
  width=\linewidth-\inwd,
  enhanced,
  boxrule=0.4pt,
  boxsep=0pt,
  left=2pt,
  top=0pt,
  colback=light-gray,
  listing only,
  top=0pt,
  bottom=0pt,
  arc=1pt,
  overlay={
    \node[
      anchor=north east,
      text width=\inwd,
      font=\footnotesize\ttfamily\color{blue!50!black},
      inner ysep=2.5mm,
      inner xsep=0pt,
      outer sep=0pt
      ] 
      at (frame.north west)
      {\stepcounter{ipythcntr}In [#1]:};
  }
  listing style=Python,
  listing options={
    basicstyle=\scriptsize\ttfamily\color{black},
    language=Python,
    escapechar=£,
    showstringspaces=false,
    commentstyle=\color{cogreen},
    keywordstyle=\bfseries\color{kwgreen},
    stringstyle=\color{darkred},
    numberstyle=\color{kwgreen},
    identifierstyle=\color{black},
    %emph={from,import,as},          % Custom highlighting
    otherkeywords={0,1,2,3,4,5,6,7,8,9,\*,==,<=,>=,+,-,\%},
    emph={*,==,<=,>=,+,-,\%},    
    emphstyle=\color{equalsign},
    extendedchars=true,
  },
}
\newtcblisting{ipythonnbout}[1][\theipythcntr]{
  enlarge left by=\inwd,
  width=\linewidth-\inwd,
  enhanced,
  boxrule=0pt,
  boxsep=0pt,
  left=2pt,
  top=0pt,
  colback=white,
  listing only,
  top=0pt,
  bottom=0pt,
  frame hidden,
  overlay={
    \node[
      anchor=north east,
      text width=\inwd,
      font=\footnotesize\ttfamily\color{red},
      inner ysep=2.5mm,
      inner xsep=0pt,
      outer sep=0pt
      ] 
      at (frame.north west)
      {\stepcounter{ipythcntr}Out [#1]:};
  }
  listing style=Python,
  listing options={
    basicstyle=\scriptsize\ttfamily\color{black},
    language=Python,
    escapechar=£,
    showstringspaces=false,
    commentstyle=\color{cogreen},
    keywordstyle=\bfseries\color{kwgreen},
    stringstyle=\color{darkred},
    numberstyle=\color{kwgreen},
    identifierstyle=\color{black},
    %emph={from,import,as},          % Custom highlighting
%    otherkeywords={as,0,1,2,3,4,5,6,7,8,9,\*,==,<=,>=,+,-},
%    emph={*,==,<=,>=,+,-},    
%    emphstyle=\color{equalsign},
    extendedchars=true,
    %literate={#}{{\#}},
  },
}
\newtcblisting{ipythonnboutno}{
  enlarge left by=\inwd,
  width=\linewidth-\inwd,
  enhanced,
  boxrule=0pt,
  boxsep=0pt,
  left=2pt,
  top=0pt,
  colback=white,
  listing only,
  top=0pt,
  bottom=0pt,
  frame hidden,
  overlay={
    \node[
      anchor=north east,
      text width=\inwd,
      font=\footnotesize\ttfamily\color{red},
      inner ysep=2.5mm,
      inner xsep=0pt,
      outer sep=0pt
      ] 
      at (frame.north west)
      {};
  }
  listing style=Python,
  listing options={
    basicstyle=\scriptsize\ttfamily\color{black},
    language=Python,
    escapechar=£,
    showstringspaces=false,
    commentstyle=\color{cogreen},
    %keywordstyle=\bfseries\color{kwgreen},
    stringstyle=\color{darkred},
    numberstyle=\color{kwgreen},
    identifierstyle=\color{black},
    %emph={from,import,as},          % Custom highlighting
    %otherkeywords={as,0,1,2,3,4,5,6,7,8,9,\[,\]},
    %emph={[,]},    
    %emphstyle=\color{black},
    extendedchars=true,
    %literate={#}{{\#}},
  },
}

% Colours
\definecolor{darkblue}{rgb}{0,0,.75}
\definecolor{light-gray}{rgb}{0.96,0.96,0.96}
\definecolor{kwgreen}{rgb}{0,0.5,0}
\definecolor{cogreen}{rgb}{0.25,0.5,0.5}
\definecolor{mygreen}{rgb}{0,0.5,0.125}
\definecolor{equalsign}{rgb}{0.66,0.13,1}
\definecolor{darkred}{rgb}{0.75,0.16,0.37}
\definecolor{dark-gray}{gray}{0.3}
\definecolor{dkgray}{rgb}{.4,.4,.4}
\definecolor{dkblue}{rgb}{0,0,.5}
\definecolor{medblue}{rgb}{0,0,.75}
\definecolor{rust}{rgb}{0.5,0.1,0.1}

%=================================================
% Paths
%=================================================

\graphicspath{{figures/}}

%=================================================
% Formatting
%=================================================

%%% Further font changes
\newcommand{\lang}{\textit}
\newcommand{\titl}{\textsl}
\newcommand{\term}{\emph}

%%% Equation numbering
\numberwithin{equation}{section} 

%%% Typesetting
\providecommand{\mathbold}[1]{\bm{#1}}  % Must be after 'fourier'
                                % package loads
%%% Annotations
\newcommand{\notate}[1]{\textcolor{red}{\textbf{[#1]}}}

%=================================================
% Theorem environment
%=================================================

\newtheorem{bigthm}{Theorem}
\renewcommand{\thebigthm}{\Roman{bigthm}}

\newtheorem{theorem}{Theorem}[section]
\newtheorem{lemma}[theorem]{Lemma}
\newtheorem{sublemma}[theorem]{Sublemma}
\newtheorem{proposition}[theorem]{Proposition}
\newtheorem{fact}[theorem]{Fact}
\newtheorem{result}[theorem]{Result}
\newtheorem{conjecture}[theorem]{Conjecture}
\newtheorem{corollary}[theorem]{Corollary}
\newtheorem{claim}[theorem]{Claim}

\theoremstyle{definition}

\newtheorem{notation}[theorem]{Notation}
\newtheorem{definition}[theorem]{Definition}
\newtheorem{example}[theorem]{Example}
\newtheorem{remark}[theorem]{Remark}
\newtheorem{assumption}[theorem]{Assumption}

\newenvironment{mainthm}{\par\textsc{Main theorem.}\it}{\par}
\renewcommand{\thebigthm}{\Alph{bigthm}}

%=================================================
% Symbols
%=================================================

% From Paul Lezeau

\newcommand{\cE}{\mathcal{E}}
\newcommand{\cN}{\mathcal{N}}
\newcommand{\cD}{\mathcal{D}}
\newcommand{\cC}{\mathscr{C}}
\newcommand{\cU}{\mathcal{U}}
\newcommand{\boldf}{\mathbf{f}}
\newcommand{\boldq}{\mathbf{q}}
\newcommand{\dadv}[2]{\Delta_\mathbf{adv}(#1, #2)}
\newcommand{\dunif}[3]{\Delta_{\mathbf{unif}, #1 } (#2,#3)}
\newcommand{\rhoadv}[1]{\rho_\mathbf{adv}(#1)}
\newcommand{\rhounif}[2]{\rho_{\mathbf{unif}, #1} (#2)}

%\newcommand{\norm}[1]{\vert \vert #1 \vert \vert}


%%% Old symbols with new names
\newcommand{\oldphi}{\phi}
\renewcommand{\phi}{\varphi}

\newcommand{\eps}{\varepsilon}
\newcommand{\e}{\varepsilon}

\newcommand{\oldmid}{\mid}
\renewcommand{\mid}{\mathrel{|}} 

%%% New symbols
\DeclareMathOperator{\md}{\mathrm{mdeg}}
\DeclareMathOperator{\pool}{\mathcal{A}}
\DeclareMathOperator{\reg}{\mathrm{reg}}
\DeclareMathOperator*{\prob}{\mathbb{P}}
\newcommand{\defby}{\overset{\mathrm{\scriptscriptstyle{def}}}{=}}
\newcommand{\half}{\tfrac{1}{2}}
\newcommand{\third}{\tfrac{1}{3}}

\newcommand{\sumnl}{\sum\nolimits}

\newcommand{\defeq}{\ensuremath{\mathrel{\mathop{:}}=}} % Definition-equals
\newcommand{\eqdef}{\ensuremath{=\mathrel{\mathop{:}}}} % Equals-definition

%%% Vector and matrix operators
\newcommand{\vct}[1]{\mathbold{#1}}
\newcommand{\mtx}[1]{\mathbold{#1}}
\newcommand{\Eye}{\mathbf{I}}
\newcommand{\binomgr}[2]{\left[\begin{array}{c} #1\\#2\end{array}\right]}

\newcommand{\transp}[1]{#1^{T}}
\newcommand{\trans}{\top}
\newcommand{\adj}{*}
\newcommand{\psinv}{\dagger}

\newcommand{\lspan}[1]{\operatorname{span}{#1}}

\newcommand{\range}{\operatorname{range}}
\newcommand{\colspan}{\operatorname{colspan}}

%\newcommand{\rank}{\operatorname{rank}}
\newcommand{\nullity}{\operatorname{null}}
%\newcommand{\ker}{\operatorname{ker}}
\newcommand{\im}{\operatorname{im}}
%\newcommand{\span}{\operatorname{span}}

%\newcommand{\diag}{\operatorname{diag}}
\newcommand{\trace}{\operatorname{tr}}

\newcommand{\codim}{\mathrm{codim}}

\newcommand{\supp}[1]{\operatorname{supp}(#1)}
\newcommand{\sign}[1]{\operatorname{sign}(#1)}

\newcommand{\smax}{\sigma_{\max}}
\newcommand{\smin}{\sigma_{\min}}

\newcommand{\nnz}{\operatorname{nnz}}
%\renewcommand{\vec}{\operatorname{vec}}

\newcommand{\Proj}{\ensuremath{\mtx{\Pi}}} % Projection operator

%%% Constants
\newcommand{\cnst}[1]{\mathrm{#1}} 
\newcommand{\econst}{\mathrm{e}}
\newcommand{\iunit}{\mathrm{i}}

\newcommand{\onevct}{\vct{e}} % All ones vector
\newcommand{\zerovct}{\vct{0}} % Zero vector

\newcommand{\Id}{\mathbf{I}}
\newcommand{\onemtx}{\mtx{1}}
\newcommand{\zeromtx}{\mtx{0}}

%%% Sets
\newcommand{\coll}[1]{\mathscr{#1}}
\newcommand{\sphere}[1]{S^{#1}}
\newcommand{\ball}[1]{B^{#1}}
\newcommand{\Grass}[2]{\mathbb{G}(#1,#2)}
\providecommand{\mathbbm}{\mathbb} % In case we don't load bbm
\newcommand{\Rplus}{\mathbbm{R}_{+}}
\newcommand{\R}{\mathbbm{R}}
\newcommand{\RP}{\mathbbm{RP}}
\newcommand{\Z}{\mathbbm{Z}}
\newcommand{\C}{\mathbbm{C}}
\newcommand{\N}{\mathbbm{N}}
\newcommand{\FF}{\mathbbm{F}}
\newcommand{\struct}{\mathcal{S}}

% Group theory
\newcommand{\stab}{\mathrm{stab}}

% Algebra
\newcommand{\End}{\mathrm{End}}
\newcommand{\Hom}{\mathrm{Hom}}
\newcommand{\Mult}{\mathrm{Mult}}

% Set operations
\newcommand{\polar}{\circ}
\newcommand{\closure}{\overline}
\newcommand{\prtensor}{\,\hat{\otimes}\,}

%%% Real and complex analysis
\newcommand{\abs}[1]{\left\vert {#1} \right\vert}
\newcommand{\abssq}[1]{{\abs{#1}}^2}

\newcommand{\sgn}[1]{\operatorname{sgn}{#1}}
\newcommand{\real}{\operatorname{Re}}
\newcommand{\imag}{\operatorname{Im}}
\newcommand{\pos}{\operatorname{Pos}}
\newcommand{\shrink}{\operatorname{Shrink}}

\newcommand{\diff}[1]{\mathrm{d}{#1}}
\newcommand{\idiff}[1]{\, \diff{#1}}

\newcommand{\gradd}{\mathrm{grad }} % Conflicts w/SIAM styles?
\newcommand{\divv}{\mathrm{div }}
\newcommand{\subdiff}{\partial}

%%% Optimization

\newcommand{\minimize}{\operatorname*{minimize}}
\newcommand{\maximize}{\operatorname*{maximize}}
\newcommand{\subjto}{\quad\text{subject to}\quad}
\newcommand{\find}{\text{find}\quad}
\newcommand{\suchthat}{\quad\text{such that}\quad}

\newcommand{\argmin}{\operatorname*{arg\; min}}
\newcommand{\argmax}{\operatorname*{arg\; max}}
\newcommand{\dom}{\mathrm{dom} }

%%% Probability & measure

\newcommand{\Prob}{\mathbbm{P}}
\newcommand{\Probe}[1]{\Prob\left({#1}\right)}
\newcommand{\Expect}{\operatorname{\mathbb{E}}}

\newcommand{\normal}{\mathcal{N}}
\newcommand{\uniform}{\textsc{Uniform}}
\newcommand{\erf}{\operatorname{erf}}
\newcommand{\homeq}{\stackrel{\partial}{\simeq}}

\DeclareMathOperator{\rvol}{rvol}
\DeclareMathOperator{\Var}{Var}
\DeclareMathOperator{\Gl}{Gl}
\newcommand{\diag}{\operatorname{diag}}

%%% Semidefinite orders
\newcommand{\psdle}{\preccurlyeq}
\newcommand{\psdge}{\succcurlyeq}

\newcommand{\psdlt}{\prec}
\newcommand{\psdgt}{\succ}

%%% Mensuration: inner products and norms

% TeX does not like either \newcommand or \renewcommand for these
% two macros.  There is probably a good reason not to use them via
% \def, but I don't know it.  
%\newcommand{\<}{\langle} 
%\newcommand{\>}{\rangle}
\newcommand{\ip}[2]{\langle {#1}, {#2} \rangle}
% --deleted the \left and \right-- Dennis
% \newcommand{\ip}[2]{\left\langle {#1},\ {#2} \right\rangle}
\newcommand{\absip}[2]{\abs{\ip{#1}{#2}}}
\newcommand{\abssqip}[2]{\abssq{\ip{#1}{#2}}}
\newcommand{\tworealip}[2]{2 \, \real{\ip{#1}{#2}}}

%\newcommand{\norm}[1]{\left\Vert {#1} \right\Vert}
\newcommand{\norm}[1]{\Vert {#1}\Vert}
\newcommand{\normsq}[1]{\norm{#1}^2}

\newcommand{\lone}[1]{\norm{#1}_{\ell_1}}
\newcommand{\smlone}[1]{\|#1\|_{\ell_1}}
\newcommand{\linf}[1]{\norm{#1}_{\ell_\infty}}
\newcommand{\sone}[1]{\norm{#1}_{S_1}}
\newcommand{\snorm}[1]{\sone{#1}}
\newcommand{\wnnorm}[1]{\norm{#1}_{*,\vct{w}}}
\newcommand{\wnnormd}[1]{\norm{#1}^*_{\vct{w}}}
\DeclareMathOperator{\dist}{dist}

% Fixed-size inner products and norms are useful sometimes
\newcommand{\smip}[2]{\bigl\langle {#1}, \ {#2} \bigr\rangle}
\newcommand{\smabsip}[2]{\bigl\vert \smip{#1}{#2} \bigr\vert}
\newcommand{\smnorm}[2]{{\bigl\Vert {#2} \bigr\Vert}_{#1}}

% Specific norms that are used frequently
\newcommand{\enormdangle}{{\ell_2}}
\newcommand{\enorm}[1]{\norm{#1}}
\newcommand{\enormsm}[1]{\enorm{\smash{#1}}}

\newcommand{\enormsq}[1]{\enorm{#1}^2}

\newcommand{\fnorm}[1]{\norm{#1}_{\mathrm{F}}}
\newcommand{\fnormsq}[1]{\fnorm{#1}^2}

\newcommand{\pnorm}[2]{\norm{#2}_{#1}}
%\newcommand{\snorm}[1]{\norm{#1}_*}

\newcommand{\triplenorm}[1]{\left\vert\!\left\vert\!\left\vert {#1} \right\vert\!\right\vert\!\right\vert} 

\newcommand{\flag}[2]{{#1 \brack #2}}

% Special cones
\newcommand{\Feas}{\mathcal{F}}
\newcommand{\Desc}{\mathcal{D}}

\newcommand{\sdim}{\delta}
\newcommand{\sdimw}{\delta^*}
% \newcommand{\sdimw}{\tilde{\delta}}
\newcommand{\ddt}[1]{\dot{#1}}
\DeclareMathOperator{\Circ}{Circ}

%%% Differential geometry
\newcommand{\Cinf}{C^{\infty}}
\newcommand{\Tensf}{\mathcal{T}}
\newcommand{\Vbundle}{\mathcal{X}}
\newcommand{\Christ}[3]{\Gamma_{{#1}{#2}}^{#3}}

% Stuff to be sorted in somewhere
\DeclareMathOperator{\Lip}{Lip}
\newcommand{\IR}{\mathbbm{R}}
\newcommand{\veps}{\varepsilon}
\newcommand{\mA}{\mathcal{A}}
\newcommand{\mB}{\mathcal{B}}
\newcommand{\mC}{\mathcal{C}}
\newcommand{\mD}{\mathcal{D}}
\newcommand{\mE}{\mathcal{E}}
\newcommand{\mF}{\mathcal{F}}
\newcommand{\mG}{\mathcal{G}}
\newcommand{\mH}{\mathcal{H}}
\newcommand{\mK}{\mathcal{K}}
\newcommand{\hR}{\hat{R}}
\newcommand{\mI}{\mathcal{I}}
\newcommand{\mM}{\mathcal{M}}
\newcommand{\mN}{\mathcal{N}}
\newcommand{\mR}{\mathcal{R}}
\newcommand{\mP}{\mathcal{P}}
\newcommand{\mL}{\mathcal{L}}
\newcommand{\mU}{\mathcal{U}}
\newcommand{\mV}{\mathcal{V}}
\newcommand{\mX}{\mathcal{X}}
\newcommand{\mY}{\mathcal{Y}}
\newcommand{\mZ}{\mathcal{Z}}
\newcommand{\bface}{\overline{\mathcal{F}}}
\newcommand{\face}{\mathcal{F}}
\newcommand{\relint}{\operatorname{relint}}
\newcommand{\relcl}{\operatorname{relcl}}
\newcommand{\cone}{\operatorname{cone}}
\newcommand{\lin}{\operatorname{lin}}
\newcommand{\Gr}{\operatorname{Gr}}
\newcommand{\powerset}{\mathscr{P}}
\newcommand{\hd}{{\operatorname{hd}}}
\newcommand{\inter}{{\operatorname{int}}}
\newcommand{\conv}{\operatorname{conv}}
\newcommand{\clconv}{\overline{\conv}}
\newcommand{\cl}{\operatorname{cl}}
\newcommand{\bd}{\operatorname{bd}}
\newcommand{\GL}{\operatorname{GL}}

% \newcommand{\tb}{\stackrel{0}{\to}}
\newcommand{\tb}{\,\,\mathring{\to}\,\,}
% \newcommand{\tdash}{\,\,\overset{e}{\to}\,\,}
% \newcommand{\tdash}{\,\,\overset{a}{\to}\,\,}
\newcommand{\tdash}{\overset{a}{\to}}

\newcommand{\K}{\mathcal{K}}
\newcommand{\B}{\mathcal{B}}
\newcommand{\F}{\mathcal{F}}
\newcommand{\D}{\mathcal{D}}
\renewcommand{\P}{\mathcal{P}}
\DeclareMathOperator{\id}{id}
\DeclareMathOperator{\gr}{gr}
\newcommand{\vp}{\varphi}
\newcommand{\maxp}{{\max}_p}
\DeclareMathOperator{\spa}{span}
\DeclareMathOperator{\vol}{vol}
\DeclareMathOperator{\ima}{im}
\DeclareMathOperator{\rk}{rk}

\newcommand{\Ren}{\mathcal{R}}
\newcommand{\T}{\mathcal{T}}
% ADJUST THE FOLLOWING
% \newcommand{\resTMP}[2]{#1,#2}
\newcommand{\resTMP}[2]{#1\to#2}
% 
\newcommand{\res}[3]{\vct{#1}_{\resTMP{#2}{#3}}}
\newcommand{\nres}[3]{\|\vct{#1}\|_{\resTMP{#2}{#3}}}
\newcommand{\sres}[3]{\sigma_{\resTMP{#2}{#3}}(\vct{#1})}
\newcommand{\csres}[3]{\overline{\sigma}_{\resTMP{#2}{#3}}(\vct{#1})}
\newcommand{\kres}[3]{\kappa_{\resTMP{#2}{#3}}(\vct{#1})}
% transpose
\newcommand{\rest}[3]{\vct{#1}^T_{\resTMP{#2}{#3}}}
\newcommand{\nrest}[3]{\|\vct{#1}^T\|_{\resTMP{#2}{#3}}}
\newcommand{\srest}[3]{\sigma_{\resTMP{#2}{#3}}(\vct{#1}^T)}
\newcommand{\csrest}[3]{\overline{\sigma}_{\resTMP{#2}{#3}}(\vct{#1}^T)}
\newcommand{\srestm}[3]{\sigma_{\resTMP{#2}{#3}}(-\vct{#1}^T)}
\newcommand{\krest}[3]{\kappa_{\resTMP{#2}{#3}}(\vct{#1}^T)}
\newcommand{\nresdag}[3]{\|\vct{#1}^\dagger\|_{\resTMP{#2}{#3}}}

\newcommand{\RCD}[3]{\Ren_{#2,#3}(\vct{#1})}
% \newcommand{\RCD}[3]{\Ren_{\resTMP{#2}{#3}}(\vct{#1})}

\newcommand{\DA}{\vct{\Delta A}}

%
\newcommand{\Rmm}{R_{\mathrm{mm}}}

\newcommand{\rbinom}[2]{\genfrac{[}{]}{0pt}{}{#1}{#2}}
\newcommand{\trbinom}[2]{{\textstyle\genfrac{[}{]}{0pt}{}{#1}{#2}}}

\newcommand{\comm}[1]{\textcolor{red}{\textbf{[#1]}}}
\newcommand{\rant}[1]{\textcolor{blue}{\textbf{[#1]}}}
%

\newcommand{\llangle}{\langle\hspace{-1.5pt}\langle}  % this is actually contained in mnsymbol, but loading this package causes trouble...
\newcommand{\rrangle}{\rangle\hspace{-1.5pt}\rangle}  % this is actually contained in mnsymbol, but loading this package causes trouble...

\newcommand{\Res}{\mathrm{Res}}
\newcommand{\Oh}{\mathcal{O}}
\newcommand{\Mbar}{\overline{M}}
\newcommand{\second}{\mathrm{II}}
\newcommand{\nablabar}{\overline{\nabla}}
\newcommand{\expm}{\mathrm{exp}}
\newcommand{\Rc}{\mathrm{Rc}}
\newcommand{\NJ}{\mathrm{NJ}}
\DeclarePairedDelimiterX{\infdivx}[2]{(}{)}{%
  #1\;\delimsize\|\;#2%
}
\newcommand{\KL}{D_{\mathrm{KL}}\infdivx}
%\newcommand{\KL}[2]{D_{\mathrm{KL}}(#1 \ || \ #2}
\newcommand{\ELBO}{\mathrm{ELBO}}
% %\newcommand{\GL}[1]{\mathrm{GL}(#1)}

\newcommand{\cover}{\lessdot}

% Semidefinite programming
\newcommand{\Sym}{\mathcal{S}}
\newcommand{\SDP}{\mathcal{S}_+}
\newcommand{\psd}{\succeq}
\newcommand{\bdot}{\bullet}

% Machine learning
\newcommand{\emprad}{\hat{\mathcal{R}}}
\newcommand{\radem}{\mathcal{R}}
\newcommand{\VCdim}{\mathrm{VC}}